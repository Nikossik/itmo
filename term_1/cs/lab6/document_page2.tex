\pagestyle{fancy}
\fancyhf{}
\fancyfoot[L]{\textbf{44}}
\renewcommand{\arraystretch}{2.5}
\begin{multicols}{2}
	\hspace{-20px}
	\begin{tabular}{|>{\centering\arraybackslash}p{35 pt}|>{\centering\arraybackslash}p{35 pt}|>{\centering\arraybackslash}p{35 pt}|>{\centering\arraybackslash}p{35 pt}|>{\centering\arraybackslash}p{35 pt}|}
		\hline\fontsize{7}{5}
		12&13&14&15&23\\
		\hline
		24&25&34&35&45\\
		\hline
		123&124&125&134&135\\
		\hline
		145&234&235&245&345\\
		\hline
	\end{tabular}\\\\
	
	\hspace{-15px}\textit{Рис. 1.}\\\\
	этому
	\[-\frac{1}{3}\sum M_{ijkl} + \frac{2}{3}\sum M_{12345} \leqslant 0.\]
	
	Теперь уже легко получить требуемый ответ. Из (7) следует, что\\\\
	\(\sum M_{ij} \geqslant 2\sum M_i - 3M\geqslant 2(5\centerdot\frac{1}{2})-3=2\)
	\columnbreak
	
	\hspace{-15px}\textbf{Формулировка общей задачи;}\\
	\textbf{случай двух зарплат}
	
	\textit{На кафтане \(M\) площади 1 имеется \(n\) заплат \(M_1\),\(M_2\), . . ., \(M_n\), площадь каждой из которых не меньше известного нам числа a; требуется оценить площадь наибольшего из пересечений \(M_ij\) заплат.} 
	
	Другими словами, для каждой конфигурации из \(n\) заплат на кафтане мы находим м а к с и м а л ь н о е по площади пересечение \(M_ij\), а потом отыскиваем м и н и м у м этого максимума \(M_ij\) по всем возможным конфигурациям заплат*). Такого рода <<минимаксные>> (то есть связанные с нахождением минимума некоторых максимумов) задачи играют в современной математике очень большую роль.
	
	Искомое число \(min max M_ij\) зависит, разумеется от заданного числа \(\alpha\), то есть является функцией от \(\alpha\); так как оно зависит также и от числа \(n\) заплат, то мы обозначим эту функцию через \(f_n(\alpha)\) (где, очевидно, \(0\leqslant \alpha \leqslant 1\), а \(n \geqslant 2\)). Решение задачи M185 сводится к доказательству равенства
\end{multicols}
